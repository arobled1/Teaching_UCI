\documentclass{article}
\usepackage[utf8]{inputenc}

\title{Chem-131C-Lec16}

\author{swflynn}
\date{May 2017}

\usepackage{natbib}
\usepackage{graphicx}
\usepackage{braket}
\usepackage{amsmath}
\usepackage[margin=0.7in]{geometry}
\usepackage{subfigure}
\usepackage{url}
\usepackage{float}

\begin{document}

\maketitle

\section*{Lecture 16; 5/8/17}
Note; next lecture (5/10/17) will be a midterm review, and the exam will be (5/12/17).
Therefore lecture notes will begin again on (5/15/17).

We are now going to revisit the Second Law of Thermodynamics from a different perspective. 
Recall, the difference between a reversible and irreversible processes is captured in the entropy change associated with these two processes. 
\begin{equation}
    \begin{split}
        dS_{rev} &= \frac{\delta q_{rev}}{T} \\
        dS_{irrev} &\geq \frac{\delta q}{T} \\
    \end{split}
\end{equation}
Consider an isothermal expansion of an ideal gas. 
If it is done reversibly heat must enter the system to keep the temperature constant during the expansion, if we consider n=1, and we double our initial volume, than w = -RTln(2), q = RTln(2). 
The process if done reversibly has $\Delta$S$_{univ}$ = 0, therefore we can write
\begin{equation}
    \Delta S_{univ} = \Delta S + \Delta S_{surr} =  0 = R\ln(2) + \frac{-q}{T} \implies R\ln(2) = \frac{q}{T}
\end{equation}
So the reversible case has a simple expression
\begin{equation}
    \Delta S = \frac{q}{T}
\end{equation}

\subsubsection*{Free Expansion}
Now Consider something irreversible like Free Expansion (free expansion is extremely irreversible). 
We know there is no work done on/by the gas during the expansion (by definition of free expansion). 
If the gas is ideal then the internal energy is not a function of volume, therefore $\Delta$U = 0 and subsequently q=0 from the first law. 
If we double the volume of the container (n=1) than the entropy is found by choosing a reversible process and calculating the heat of that process to enter into the second law.
An isothermal expansion can be used to model our reversible heat transfer for the system, and we find $\Delta$S = Rln(2). 
Because our free expansion has no heat exchange with the environment, the entropy of the surroundings does not change and we get an overall entropy increase of the universe by Rln(2).
Again, free expansion is a limiting case of extremely irreversible, we must increase the entropy of the universe for this process. 

\subsection*{Clausius Inequality}
We know that entropy is a state function, and using an integrating factor we found. 
\begin{equation}
    dS \geq \frac{\delta q}{T}
\end{equation}
Being a state state function we know an integral around a closed loop must give an area of 0.
\begin{equation}
    \oint dS = 0 \implies 0 \geq \frac{\delta q}{T}
\end{equation}
This relationship is  another statement of the second law (usually used engineering thermodynamics) known as the \textbf{Clausius Inequality}. 
\begin{equation}
    0 \geq \oint \frac{\delta q}{T}
\end{equation}
Where the equality is generated if every step in the process is done reversibly. 

\subsection*{Helmholtz Free Energy}
Consider the first law
\begin{equation}
    dU = \delta q + \delta w
\end{equation}
If we consider a closed system with only PV work, and keep volume constant (rigid container) we can write
\begin{equation}
    dU = \delta q
\end{equation}
This means we can write the second law for a closed system and constant volume as
\begin{equation}
\begin{split}
    dS &\geq \left(\frac{dU}{T}\right)_V\\
    TdS &\geq dU \\
    0 &\geq dU - TdS
    \end{split}
\end{equation}
If the temperature is also constant we can pull it into the differential and write. 
\begin{equation}
d(U-TS) \leq 0
\end{equation}
Which is true for a constant T,V process, and we see the process must be negative to be spontaneous. 

From this analysis is seems useful to define a new thermodynamic potential 
\begin{equation}
A \equiv U - TS
\end{equation}
A is known as the Helmholtz Free Energy (often labeled as F too), and a negative Helmholtz implies a spontaneous process. 
Essentially this potential is useful because it 'tracks' the  necessary increase in the entropy of the universe during a process through monitoring the environment. 
If we consider a constant temperature process we have
\begin{equation}
\Delta A = \Delta U - T\Delta S
\end{equation}
We know that $\Delta$A must be negative to be spontaneous and you can compensate through negative internal energy or positive entropy. 
A negative $\Delta$A implies energy enters the environment which will increase its entropy. 

If we do a process reversibly we get our equality
\begin{equation}
    \Delta S_r = \frac{q_r}{T}
\end{equation}
We can use this to consider the reversible Helmholtz Equation. 
\begin{equation}
    \begin{split}
        \Delta A_r &= \Delta U_r - T\Delta S_r \\
        &= \Delta U_r - T\frac{q_r}{T} \\
        &= \Delta U_r - q_r \\
        &= q_r + w_r - q_r \\
        \Delta A_r &= w_r
    \end{split}
\end{equation}
So we see the Helmholtz free energy with constant Temperature done reversibly is equal to the reversible work of the process. 

\subsubsection*{Gibbs Free Energy}
Experimentalists would really like a thermodynamic potential with characteristic variables T and P. 
This way working in the lab they could easily control all of the variables. 
Consider a constant pressure process, we could write 
\begin{equation}
    \delta q = dH = C_pdT
\end{equation}

Heat capacity says it takes some amount of heat to change temperature; for an ideal gas we have
\begin{equation}
    C_v = \frac{3}{2}R \qquad C_p = \frac{5}{2}R
\end{equation}
So we see that we need more heat to get the same temperature change for a constant pressure process.
This makes complete sense!
The volume changes, so when you heat the gas some of the energy goes into the volume change not just the temperature change. 
And when we change the volume we must be doing work which takes energy. 

So a constant pressure process let's us write $\delta$q = dH, and then assuming a constant temperature we can write dS - $\frac{\delta q}{Y} \geq$ 0 $\implies$ dH - TdS $\leq$ 0. 
So our final relationship is that d(H-TS)$\leq$ 0. 
\begin{equation}
    G \equiv H - TS \qquad dG \leq 0
\end{equation}
So the Gibss Free energy is a natural variable of G(P,T) and is negative for a spontaneous process. 
Where again the direction of spontaneity is ultimately due to the 2$^{nd}$ Law. 

\subsection*{Legendre Transforms}
Legendre Transforms are the mathematical operation for concerting functions of some set of variables to another set of variables. 
The method is very simple, we know that all thermodynamic potentials are exact differentials, this allows us to write two different equations.
The first comes from our fundamental equations, and the second from a total differential of the thermodynamic potential. 
\subsubsection*{Internal Energy}
Consider U(S,V) from the fundamental equation (combined first and second law). 
\begin{equation}
    \begin{split}
        dU &= TdS - PdV \\
        dU &= \left(\frac{\partial U}{\partial S}\right)_VdS + \left(\frac{\partial U}{\partial V}\right)_SdV 
    \end{split}
\end{equation}
We know each of these statements must be true (the last line is the total differential for U(S,V) which we see from the fundamental equation), we can therefore equate variables and write down some new definitions. 
\begin{equation}
    T = \left(\frac{\partial U}{\partial S}\right)_V \quad \text{and} \quad -P = \left(\frac{\partial U}{\partial V}\right)_S
\end{equation}
We can now simply repeat this process for all of our Thermodynamic Potential definitions. 

\subsubsection*{Enthalpy}
Consider the Enthalpy.
\begin{equation}
    \begin{split}
    H &\equiv U + PV \\
     dH &= dU + d(PV) \\
         &= dU +PdV + VdP\\
         &= TdS - PdV + PdV + VdP\\
        dH &= TdS + VdP  \qquad \xrightarrow{H(S,P)} \\
        dH &= \left(\frac{\partial H}{\partial S}\right)_P dS + \left(\frac{\partial H}{\partial P}\right)_S dP
    \end{split}
\end{equation}
\begin{equation*}
     T= \left(\frac{\partial H}{\partial S}\right)_P \quad \text{and} \quad  V= \left(\frac{\partial H}{\partial P}\right)_S
\end{equation*}

\subsubsection*{Helmholtz Free Energy}
Consider the Helmholtz Free Energy next. 
\begin{equation}
    \begin{split}
    A &\equiv U - TS \\
        dA &= dU - d(TS) \\
          dA &= -PdV - SdT \qquad \xrightarrow{A(T,V)} \\
        dA &= \left(\frac{\partial A}{\partial V}\right)_T dV + \left(\frac{\partial A}{\partial T}\right)_V dT
    \end{split}
\end{equation}
\begin{equation*}
    -P = \left(\frac{\partial A}{\partial V}\right)_T \quad \text{and} \quad  -S= \left(\frac{\partial A}{\partial T}\right)_V
\end{equation*}

\subsubsection*{Gibbs Free energy}
Finally we can consider the Gibbs Potential.
\begin{equation}
    \begin{split}
    G &\equiv H - TS \\
        dG &= dH - d(TS) \\
          &= d(U+PV) - d(TS) \\
          dG &= -SdT + VdP \qquad \xrightarrow{G(T,P)}\\
        dG &= \left(\frac{\partial G}{\partial T}\right)_P dT + \left(\frac{\partial G}{\partial P}\right)_T dP
    \end{split}
\end{equation}
\begin{equation*}
     -S= \left(\frac{\partial G}{\partial T}\right)_P \quad \text{and} \quad V = \left(\frac{\partial G}{\partial P}\right)_T
\end{equation*}

\end{document}