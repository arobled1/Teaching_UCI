\documentclass{article}
\usepackage[utf8]{inputenc}

\title{Chem-131C-Lec22}

\author{swflynn}
\date{May 2017}

\usepackage{natbib}
\usepackage{graphicx}
\usepackage{braket}
\usepackage{amsmath}
\usepackage[margin=0.7in]{geometry}
\usepackage{subfigure}
\usepackage{url}
\usepackage{float}
\usepackage[version=3]{mhchem}

\begin{document}

\maketitle

\section*{Lecture 22 ; 5/26/17}
Note: There will be no lecture on 5/29/17 Due to Memorial Day.

Previously in Thermodynamics, we entered the fascinating world of Chemistry with the reaction 
\begin{equation}
    A \rightleftharpoons B 
\end{equation}
Because we are working in closed systems and assume conservation of mass and energy we concluded that any change in n$_A$ was directly proportional to n$_B$ and we subsequently defined the extent of reaction ($\xi$). 
\begin{equation}
dn_B = d\xi \qquad dn_A = -d\xi
\end{equation}
If we assumed our process to occur at constant Temperature and constant Pressure we wrote down the change in the Gibbs Free Energy due to the reaction as 
\begin{equation}
\Delta_rG \equiv \left(\frac{\partial G}{\partial \xi}\right)_{T,P}
\end{equation}
Which takes a very simple form in this chemical reaction as 
\begin{equation}
    \begin{split}
        \Delta_rG &= \mu_B - \mu_A = \mu_B^0 - \mu_A^0 + RT\ln\left(Q\right) \\
        Q &\equiv \frac{P_B}{P_A}
    \end{split}
\end{equation}
Here we defined a new Q the Reaction Quotient. 

\subsection*{Equilibrium}
At equilibrium there is no net change in the free energy occurring (all equilibrium are dynamic, the net change is what we consider). 
Therefore we expect $\Delta_rG = 0$ at equilibrium only. 
Because equilibrium is special we will define the \textbf{Equilibrium Constant} K.
\begin{equation}
    K_{equ} = Q_{equ} = \frac{P_B^{eq}}{P_A^{eq}}
\end{equation}

Yet again we have reference terms that we need to care about. 
We can just define a  reference Gibbs reaction energy, which occurs at our reference pressures; P$_A$ = P$^0$ and  P$_B$ = P$^0$, $\Delta_rG^0$. 
Recall our sign convention, we defined the extent of reaction to be positive for making product:  $dn_B = d\xi$ and $dn_A = -d\xi$. 

To understand which way the reaction is going consider 
\begin{equation}
    \Delta_rG = \Delta_rG^0 + RT\ln(Q)
\end{equation}
The reference is just some number so look at the pressure ratio to determine which direction the reaction will spontaneously travel to reach equilibrium. 
% 
$\Delta_rG < 0 \implies A \rightarrow B$
%

At equilibrium we know there is no change in the free energy occurring
\begin{equation}
\begin{split}
    \Delta_rG = 0 &= \Delta_rG^0 + RT\ln(K)\\
    K = e^{\frac{-\Delta_rG^0}{RT}}
    \end{split}
\end{equation}
So at equilibrium we can write the equilibrium coefficient in terms of the standard state Gibbs Free Energy. 

\subsection*{Generalization}
We now want to generalize this strategy to an arbitrary reaction
\begin{equation}
aA + bB + \cdots \rightleftharpoons cC + dD + \cdots
\end{equation}
In compressed notation we could write this same statement as 
\begin{equation}
\sum_i \gamma_iJ_i = 0
\end{equation}
Where we arbitrarily choose to set the $\gamma$ of reactants to be negative. 

\subsubsection*{Example}
\begin{equation}
\begin{split}
2A + 3B &\rightleftharpoons C + 2D \\
C + 2D - &2A + 3B = 0 \\
\gamma_c = 1 \quad \gamma_D = 2 \quad &\gamma_A = -2 \quad \gamma_B = -3
\end{split}
\end{equation}

\subsection*{Constraints}
If we consider this reaction occurring at constant T and P we can write our total differential as
\begin{equation}
dG = \sum_i \mu_idn_i
\end{equation}
But just as before the total number of moles for each species is not needed (it is more information than we need in a closed system). 
We say the moles of each species are not independent, and we again invoke the extent of reaction.  
\begin{equation}
dn_i = \gamma_id\xi
\end{equation}
We can combine these two equations to write our change in G wrt $\xi$. 
\begin{equation}
\begin{split}
dG &= \left(\sum_i\mu_i\gamma_i\right)d\xi \\
\Delta_rG &= \left(\frac{\partial G}{\partial \xi}
\right)_{T,P} \\
&= \sum_i \gamma_i \mu_i
\end{split}
\end{equation}
So again we just define the slope of the G, $\xi$ curve to be the change in the Gibbs Free Energy due to the reaction, and we must sum over all of our species, weighted by their stoichometry. 

\subsubsection*{Example Again}
\begin{equation}
\begin{split}
    \Delta_rG &= \sum_i \gamma_i \mu_i = \sum_i \left[\mu_i^0 + RT \ln(\frac{P_i}{P^0})\right] \quad \xrightarrow{P^0 = 1}\\
    &= \Delta_rG^0 + \sum_i \gamma_iRT\ln(P_i) \mid \Delta_rG^0 = \sum_i \gamma_i\mu_i^0\\
    &= \Delta_rG^0 + RT\ln(Q) \mid Q = \prod_i P_i^{\gamma_i}\\
   & \xrightarrow{2A + 3B \rightleftharpoons C + 2D}\\
   \Delta_rG &= \Delta_rG^0 + RT\ln(P_c) + 2RT\ln(P_D) - 2RT\ln(P_A) - 3RT\ln(P_B) \\
   &= \Delta_rG^0 + RT \ln\left(\frac{P_CP_D^2}{P_A^2P_B^3}\right)
\end{split}
\end{equation}
This result should look somewhat familiar, again at equilibrium we know the following is true
\begin{equation}
\Delta_rG = 0 = \Delta_rG^0 + RT\ln(K) \mid K = \frac{P_{C,eq}P_{D,eq}^{2}}{P_{A,eq}^{2}P_{B,eq}^{3}}
\end{equation}
And again at equilibrium we can replace the Reaction Quotient with the equilibrium values (while maintaining the stoichometry relationships). 

\subsection*{Activity Generalization}
This derivation uses the Partial Pressure of the gases during the reaction. 
This is ultimately an ideal gas assumption, a more general approach would need to use a different 'weight' for  the calculation. 
The ideal gas approximation is good for dilute gases, or weakly interacting gases, however, stronger interactions and higher concentrations will quickly deviate from ideal. 

Note a very similar derivation (and assumption) is made for liquid mixtures too (giving things like Henry's Law and Rault's Law). 

In the case of gases we would define something called \textbf{fugacity}, to capture the deviations of reality from ideality.
In the case of a liquid we would define an \textbf{activity} (a$_i$). 
So you would write down all of the same equations as before, replacing the partial pressures with a (whatever a is..) or with f (whatever fugacity is...). 
\begin{equation}
\mu_i = \mu_i^0 + RT \ln(a_i)
\end{equation}

We are probably not going to discuss any specific activity models in this course.
If you are interested the simplest model for an activity coefficient theoretically would probably be the 1 parameter Margules Equation.
Ionic liquid solutions (highly non-ideal) can use a Debye-Huckel equation to model the activity coefficient of an ion in ionic solution.
 
\subsection*{Example Problems}
Here are some example problems for working with the concepts of the last few lectures. 

\subsubsection*{Calculate the Standard Gibbs Energy of Reaction}
Consider the following chemical reaction at 298K.
\begin{equation}
    \ce{CO_{(g)} + CH3OH_{(l)} -> CH3COOH_{(l)}}
\end{equation}
To solve this problem we are asked about the standard Gibbs Energy of reaction (these are values you look up in a table to solve). 
We can break up the problem into products - reactants (based on our sign convention). 
\begin{equation}
\begin{split}
    \Delta_{rxn}G^0 &= \sum_{products} \gamma_i\Delta_fG^0 - \sum_{reactants}\gamma_i\Delta_fG^0\\
    & = \Delta_fG^0(\ce{CH3COOH_{(l)}}) - \Delta_fG^0(\ce{CO_{(g)}}) - \Delta_fG^0(\ce{CH3OH_{(l)}}) \\
    &= -389.9 (kJ/mol) - (-137.17) (kJ/mol) - (-166.27) (kJ/mol) \\
    \Delta_{rxn}G^0 &= -86.46 (kJ/mol)
    \end{split}
\end{equation}
To determine the standard Gibbs Free Energy of reaction we simply determine the Gibbs Free Energy of Formation for each component. 
This reaction is spontaneous because or Gibbs Free Energy of Reaction is negative. 

\subsubsection*{Equilibrium Coefficients}
Consider Hemoglobin, and its binding to both oxygen and carbon monoxide. 
Determine which pathway is chemically favored. 
\begin{equation}
\begin{split}
\ce{Heme + O2 <=> Heme-O2} \qquad K \approx 9\times 10^{18}\\
\ce{Heme + CO <=> Heme-CO} \qquad K \approx 2\times 10^{23}
\end{split}
\end{equation}
If we consider a simple ration of the two reactions equilibrium coefficients we find that the CO complex is highly favored at these conditions. 
\begin{equation}
K_r \approx \frac{2\times 10^{23}}{9\times 10^{18}} \approx 3\times10^4
\end{equation}

\subsubsection*{ICE Table}
Consider the reaction
\begin{equation}
\ce{Z_{(l)} + B_{(l)} <=> C_{(l)} + D_{(l)} }
\end{equation}
At 300K the equilibrium coefficient for this reaction is 10. 
\begin{enumerate}
    \item Determine the reaction quotient for this reaction with [Z] = 0.5M, [B] = 5.0M, [C] = 2.0M, [D] = 0.2M.
    \item How many moles of each gas will be present at equilibrium?
\end{enumerate}
We note that these are all liquid species, therefore our reaction quotient should be written in terms of activity coefficients.
If we assume the solution is ideal we can approximate the activity coefficients as the concentration of the chemical species, and we simply plug in values. 
\begin{equation}
Q = \frac{a_Da_C}{a_Ba_Z} \approx \frac{[D][C]}{]B][Z]} = \frac{(2.0) (0.2)}{(5.0)(0.5)} = 0.16
\end{equation}
Because this value is less than 10, we would need to generate more products to reach equilibrium (the product concentration must increase at equilibrium). 

To determine the equilibrium concentrations we can use an ICE Table (look back on your general chemistry notes). 
\begin{center}
  \begin{tabular}{ | l | c | r | c | c | }
    \hline
    ICE & Z & B & C & D \\ \hline
    I & 0.5 & 5.0 & 2.0 & 0.2 \\ \hline
    C & -x & -x & +x & +x \\ \hline
    E & 0.5-x & 5.0-x & 2.0+x & 0.2+x \\ \hline
  \end{tabular}
\end{center}
Using our equilibrium coefficient we can then determine the equilibrium amounts of reactants and products with the quadratic formula. 
\begin{equation}
\begin{split}
    K &= 10 = \frac{(2+x)(2+x)}{(5-x)(5-x)} = \frac{x^2+2.2x+.4}{x^2 - 5.5x +2.5} \\
    0 &= 9x^2 - 7.2x + 24.6 \\ 
    & x = 0.47, \qquad x = 5.89
\end{split}
\end{equation}
Given the 5.89 root is not physical (we do not have enough concentrations for this result) we know the other root must show the change in x necessary for the reaction to reach equilibrium. 

\end{document}
