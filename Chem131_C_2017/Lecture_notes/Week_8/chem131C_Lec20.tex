\documentclass{article}
\usepackage[utf8]{inputenc}

\title{Chem-131C-Lec20}

\author{swflynn}
\date{May 2017}

\usepackage{natbib}
\usepackage{graphicx}
\usepackage{braket}
\usepackage{amsmath}
\usepackage[margin=0.7in]{geometry}
\usepackage{subfigure}
\usepackage{url}
\usepackage{float}
\usepackage[version=3]{mhchem}

\begin{document}

\maketitle

\section*{Lecture 20; 5/22/17}
On Wednesday and Friday we will look at Chapter 26 (due to Dr. Martens Traveling), so we will start this lecture material, and re-visit it in a week. 

\subsection*{Mixing Gases}
Consider mixing gas A and gas B, both at T$_0$ and P$_0$, $\frac{V_0}{2}$. 
If we consider the gases to be ideal, there is no pressure gradient that occurs during the mixing process. 
If we take n$_A$ moles of A and $n_B$ moles of B we would find. 
\begin{equation}
    \begin{split}
        \mu_A &= \mu_A^0 + RT \ln\left(\frac{P_A}{P_A^0}\right)\\
         \mu_B &= \mu_B^0 + RT \ln\left(\frac{P_B}{P_B^0}\right)\\
    \end{split}
\end{equation}
Do not be confused by the reference states!
If P$_A$ = P$_A^0$ then $\mu_A = \mu_A^0$ (references can be chosen for convenience). 
For convenience let's let P$_A^0$ = P$_B^0$ = P$_0$, which is just the pressure of our system (there is no pressure change associated with mixing ideal gases). 
Therefore $\mu_A = \mu_A^0$ and $\mu_B = \mu_B^0$.

When we mix we expect there to be a decrease in the overall chemical potential.
We know these two gases will mix spontaneously (because why wouldn't they?), and we know we want to minimize chemical potential at equilibrium. 

\begin{equation}
    G = \sum_i \mu_i n_i = n_A\mu_A + n_B \mu_B
\end{equation}
We know that the final pressure will depend on how much of each atom type is in the gas phase (relative to the other components in the mixture).
\begin{equation}
    \begin{split}
        P_A^f &= \frac{n_A}{n_A+n_B}P_0 \equiv x_AP_0\\
        P_B^f &= \frac{n_B}{n_A+n_B}P_0 \equiv x_BP_0\\
    \end{split}
\end{equation}
Here x is the \textbf{mole fraction} and P$_0$ is our total pressure.
According to \textbf{Dalton's Law} our total pressure will just be the sum of our partial pressures (for ideal gases).
Using Dalton's Law we can write our standard Ideal Gas EOS as
\begin{equation}
    P_0V = (n_A+n_B)RT
\end{equation}
We can now solve for the chemical potential of the final states, after mixing as:
\begin{equation}
    \begin{split}
        \mu_A^f = \mu_A^0 + RT\ln(x_A)\\
        \mu_B^f = \mu_B^0 + RT\ln(x_B)\\
    \end{split}
\end{equation}
Now that we have the chemical potential we can calculate the Gibbs Free Energy.
\begin{equation}
    \begin{split}
        \Delta G = n_ART\ln(x_A) + n_BRT\ln(x_B) \\
        \Delta G = nRT\left[x_A\ln(x_A) + x_B\ln(x_B)\right]
    \end{split}
\end{equation}
Where I have made the substitution n$_A$ = x$_A$n and n$_B$ = x$_B$n (n refers to the total number of moles in the system). 

Because the mole fractions must be less than one we find that each component has a lower chemical potential due to mixing when compared to the un-mixed substances. 
This is consistent with our initial hypothesis, the mixing process is spontaneous because it lowers the Gibbs Free Energy!

Consider the Gibbs Free Energy of the mixing. 
\begin{equation}
    \begin{split}
        \Delta G &= \Delta H - T\Delta S \xrightarrow{\Delta H = 0 \text{mix}}\\
        &= -T\Delta S \implies \\
        \Delta S &= -nR\left[x_A\ln(x_A) + x_B\ln(x_B)\right]
    \end{split}
\end{equation}
It is interesting to note that the mixing entropy we found is very similar to our statistical mechanics mixing equation from before.
\begin{equation}
    S = -k\sum_iP_i\ln(P_i)
\end{equation}
This implies that the mixing process is entropically driven.
This should make sense, when you mix the two gases you increase the number of combinations available to all the atoms in the system. 

\subsection*{Statistical Mechanics}
Let's consider this mixing problem from the angle of Statistical Mechanics and make sure it is consistent with our thermodynamics results. 
In the land of Stat. Mech. we found that the energy expressions (A and G are energies) take the form
\begin{equation}
    E = -kT\ln(Q)
\end{equation}
Remember A(T,V) and shows what the maximum amount of work you can get out of a system. 
In our statistical mechanics lectures we used the Boltzmann Statistics approximation to say (q is the number of states, N the number of particles, valid if q 
$>>$N).
\begin{equation}
    Q \approx \frac{q^N}{N!}
\end{equation}
We then used Stirling's Approximation to write
\begin{equation}
    Q = \left(\frac{eq}{N}\right)^N
\end{equation}
We can use these same approximation to evaluate the Helmholtz Free Energy (or any Free Energy). 
\begin{equation}
\begin{split}
    A &= -kT \ln\left(\frac{eq}{N}\right)^N \\
    &= -NkT \ln\left(\frac{eq}{N}\right) = -NkT \ln\left(\frac{q}{N}\right) - NkT \\
    A &= -nRT \ln\frac{q}{N} - nRT
\end{split}
\end{equation}
If we are assuming an ideal gas, 1 component mixture, we can write G = A + PV = A + nRT, and simplify the above expression. 
\begin{equation}
G = -nRT\ln\left(\frac{q}{N}\right)
\end{equation}
We know that the Hamiltonian is separable from Quantum Mechanics.
Because the Hamiltonian is separable our Partition function will become a product of the two components (Boltzmann Statistics approximates the distributions as independent). 
\begin{equation}
    \begin{split}
        \hat{H} &= \hat{H}_A + \hat{H}_B \\
        Q &= Q_A + Q_B = \frac{q_A^{N_A}}{N_A!}\frac{q_B^{N_B}}{N_B!} = \frac{V_A}{\Lambda^3}\frac{V_B}{\Lambda^3}
    \end{split}
\end{equation}
If we have a 1-component system we can write 
\begin{equation}
    \mu = \frac{G}{n} = \frac{RT\ln(Q)}{n}
\end{equation}
The free energy is extensive, therefore the total will just be the sum of our mixture components.
\begin{equation}
    \begin{split}
        G &= G_A + G_B = n_A\mu_A + n_B\mu_B \\
        &= -n_ART\ln\left(\frac{q_A}{N_A}\right) + -n_BRT\ln\left(\frac{q_B}{N_B}\right)
    \end{split}
\end{equation}
If we now want to calculate the change in Gibbs due to mixing, we simple construct the initial and final states and take their difference. 
\begin{equation}
    \begin{split}
        G_i &= -n_ART\ln\left(\frac{V_0/2}{N_A\Lambda^3}\right) - n_BRT\ln\left(\frac{V_0/2}{N_B\Lambda^3}\right) \\
         G_f &= -n_ART\ln\left(\frac{V_0}{N_A\Lambda^3}\right) - n_BRT\ln\left(\frac{V_0}{N_B\Lambda^3}\right) \\
         \Delta G &= -n_ART\ln(2) - n_BRT\ln(2)
    \end{split}
\end{equation}
And we again get the same result as before. 

\subsection*{Gibbs Paradox}
A slight aside, the \textbf{Gibbs Paradox} is essentially a seeming paradox that occurs when you consider mixing.
If you do not account for the indistinguishable nature of the mixing process when considering atoms it can seem like you violate the second law. 
Essentially when you mix two of the same gases (exactly the same, temperature atoms, etc) you cannot identify which atom is which.
This distinguishable versus indistinguishable factor is crucial in statistical mechanics!
As far as we know the Laws of Thermodynamics (they are still only laws though) are not violated!

\end{document}