\documentclass{article}
\usepackage[utf8]{inputenc}

\title{Chem-131C-Lec19}

\author{swflynn}
\date{May 2017}

\usepackage{natbib}
\usepackage{graphicx}
\usepackage{braket}
\usepackage{amsmath}
\usepackage[margin=0.7in]{geometry}
\usepackage{subfigure}
\usepackage{url}
\usepackage{float}
\usepackage[version=3]{mhchem}

\begin{document}

\maketitle

\section*{Lecture 19 ; 5/19/17}
If we want to characterize a phase transition we could look at the sharp spikes in heat capacity that occur during the transition (the temperature does not change so our heat capacity becomes $\infty$ during the process).
When you look at a phase diagram, the lines refer to two phases in equilibrium.
This equilibrium is characterized by their chemical potentials being equal ($\mu_\alpha = \mu_\beta$). 
If  you are not in equilibrium, whatever phase dominates has the lower chemical potential. 
If $\mu_{\alpha} < \mu_{\beta}$ then $\beta$ will spontaneously become $\alpha$. 

If we are working with transitions between only two phases than the change of phase 1 must become phase 2, they are linked. 
In this scenario it is useful to define molar quantities where we divide by n (if there are more than 2 phases it is not clear what you would divide by). 
\begin{equation}
    \begin{split}
        d\mu_\alpha &= d\mu_\beta \\
        d\mu &= -SdT + VdP \xrightarrow{equilibrium}\\
        \mu_\alpha &= \mu_\beta \implies -S_\alpha dT + \bar{V}_\alpha dP =  -S_\beta dT + \bar{V}_\beta dP \implies \\
        \frac{dP}{dT} &= \frac{\Delta \bar{S}}{\Delta \bar{V}}
    \end{split}
\end{equation}
This last equation is the slope of a line on a P, T phase diagram (along this line your chemical potentials are equal). 

\subsection*{Chapter 24; Mixtures}
We want to start considering c-component chemical systems. 
Our thermodynamic potentials will need to account for all of these chemical species now. 
\begin{equation}
    \begin{split}
        U(S,V,n_1,n_2,...n_c)\\
        A(S,V,n_1,n_2,...n_c)\\
        G(S,V,n_1,n_2,...n_c)\\
        \cdots
    \end{split}
\end{equation}
A chemical reaction is at equilibrium when the total Gibbs Energy is at a minimum wrt n$_i$. 

\subsection*{Extensive Quantities}
A \textbf{Homogeneous Equation} can be written mathematically as a function that obeys
\begin{equation}
    U(\lambda S, \lambda V, \lambda n_1, ..., \lambda n_c) = \lambda U(S,V,n_1,....,n_c)
\end{equation}
In fact this is a degree = 1 Homogeneous Equation.
All of our functions are factorable wrt $\lambda$. 
Some explicit examples of Homogeneous Equations would be
f(x,y) = ax + by: degree 1. f(x,y) = ax$^2$ + bxy + cy$^2$; degree 2 (the total order of variables defines the degree in this context). 
Consider a derivative of our homogeneous equation (I am being lazy, all other variables are held constant during the partial derivatives). 
\begin{equation}
    \begin{split}
        \frac{d}{d\lambda} U(\lambda S, \lambda V, \lambda n_1, ..., \lambda n_c) &=  \frac{d}{d\lambda} \lambda U(S,V,n_1,....,n_c) \\
        \frac{\partial U}{\partial \lambda S }\frac{\partial \lambda S}{\partial \lambda} +  \frac{\partial U}{\partial \lambda V }\frac{\partial \lambda V}{\partial \lambda} +  \sum_{i=1}^c\frac{\partial U}{\partial \lambda n_i }\frac{\partial \lambda n_i}{\partial \lambda} &= U(S,V,n_1,....,n_c)\\
        S \frac{\partial U}{\partial \lambda S } +  V \frac{\partial U}{\partial \lambda V } +  n_i \sum_{i=1}^c\frac{\partial U}{\partial \lambda n_i } &= U(S,V,n_1,....,n_c) \xrightarrow{\lambda \equiv 1}\\
         S \frac{\partial U}{\partial S } +  V \frac{\partial U}{\partial V } +  n_i \sum_{i=1}^c\frac{\partial U}{\partial n_i } &= U(S,V,n_1,....,n_c)\\
         S(T) + V(-P) +\sum_{i=1}^c (n_i)\mu_i &= U \\
         U = TS - PV + \sum_i\mu_in_i
    \end{split}
\end{equation}
This last line is a very famous equation known as the \textbf{Euler Equation}. 
We also know that G = U + PV - TS, from our Fundamental Equations.
To be consistent it must be true that 
\begin{equation}
    G = \sum_{i=1}^c\mu_in_i
\end{equation}
Consider 
\begin{equation}
    dU = TdS + SdT -PdV - VdP +\sum_i(n_id\mu_i + \mu_idn_i)
\end{equation}
Before we found the fundamental equation for U to be
\begin{equation}
    dU = TdS - PdV + \sum_i\mu_idn_i
\end{equation}
Thermodynamics must be consistent, therefore
\begin{equation}
    0 = SdT -VdP \sum_{i=1}^cn_id\mu_i
\end{equation}
And this realization is known as the \textbf{Gibbs Duhem Equation}.

The Gibbs Duhem Equation tells us for a constant T,P process $\sum n_id\mu_i$ = 0.
If we had a 2-component system we can use this equation as a constraint. 
\begin{equation}
    n_1d\mu_1 + n_2 d\mu_2 = 0
\end{equation}

\subsection*{Pressure Dependence}
If we are interested in the pressure dependence of G and $\mu$. 
\begin{equation}
    \begin{split}
        dG &= -SdT + VdP \\
        dG(P) &= VdP \xrightarrow{I.G.} \\
        dG &= \frac{nRT}{P}dP \\
        G &= G_0 + nRT\ln\left(\frac{P}{P_0}\right) \\
        \mu &= \mu_0 + RT\ln\left(\frac{P}{P_0}\right)
    \end{split}
\end{equation}
The last line follows if we assume a 1-component system; the Euler Equation would read (G=n$\mu$). 

\subsection*{Free Expansion; Again}
If we consider a free expansion into vacuum (double the volume) for an ideal gas we know there is no Temperature change. 
If we have a 1 component system (n=1) then 
\begin{equation}
\begin{split}
    \mu_i &= \mu_0 +RT \ln\frac{P_i}{P_0} \\
    \mu_f &= \mu_0 +RT \ln\frac{P_f}{P_0} \\
    P_f (2V_0) &= RT_0 \implies P_f = \frac{1}{2}P_i\\
    \Delta\mu &= -RT\ln(2)
    \end{split}
\end{equation}
We could also get this same result from a Fundamental Equation, we know the entropy change for this process is Rln(2). 
\begin{equation}
    \begin{split}
        \Delta \mu = \Delta H - T\Delta S \\
        \Delta \mu = -R\ln(2)
    \end{split}
\end{equation}

\end{document}
