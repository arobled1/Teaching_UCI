\documentclass{article}
\usepackage[utf8]{inputenc}

\title{Chem-131C-Lec23}

\author{swflynn}
\date{May 2017}

\usepackage{natbib}
\usepackage{graphicx}
\usepackage{braket}
\usepackage{amsmath}
\usepackage[margin=0.7in]{geometry}
\usepackage{subfigure}
\usepackage{url}
\usepackage{float}
\usepackage[version=3]{mhchem}

\begin{document}

\maketitle

\section*{Lecture 23; 5/31/17}
In Chapter 26 we discussed a chemical reaction, the spark notes go something like....
\begin{equation}
aA + bB  \rightleftharpoons cC + dD 
\end{equation}
During a constant Temperature and Pressure process we found
\begin{equation}
    dG = \sum_i \nu_i\mu_id\xi
\end{equation}
We know that the Gibbs Free energy is minimized at Equilibrium, therefore
\begin{equation}
    dG = 0 = \sum_i\nu_i\mu_i
\end{equation}
Finally we found reference states entering into all of our equations (to make sure our logarithms don't have units).
These references are just numbers and should be chosen to make life easy!

We then wrote the chemical potential as
\begin{equation}
    \mu_i = \mu_i^0 + RT \ln\left(\frac{P_i}{P_i^0}\right)
\end{equation}
If we are interested in an ideal gas, P$_i^0$ = 1atm (the standard reference state for a gas). 

To discuss a reaction we are interested in the equilibrium coefficient K 
\begin{equation}
    K_{eq} =\prod_i P_{i,eq}^{\nu_i}
\end{equation}
For our chemical reaction above we can write. 
\begin{equation}
    K = \frac{P_c^cP_D^d}{P_A^aP_B^b} = e^{-\frac{\Delta G^0}{RT}}
\end{equation}
Looking at this we realize that an equilibrium constant greater than 1 must favor formation of products. 

\subsection*{Chemical Potential}
We started the course discussing the internal energy U(S,V), and now we have been discussing Gibbs, G(T,P). 
Think back to a particle in a box, we were considering a constant T and V process in this context, so we would want to use Helmholtz A(T,V,n). 
All of the equations will look similar to those found working with Gibbs, after-all this is just another type of thermodynamic energy. 
\begin{equation}
    dA = \sum_i\mu_udn_i = \sum_i \mu_i\nu_id\xi
\end{equation}
And at equilibrium the energy will again be minimized (dA = 0 at equilibrium). 

Note we previously defined 
\begin{equation}
    \mu_i \equiv \left(\frac{\partial G}{\partial n_i}\right)_{T,P,n_j}
\end{equation}
But it actually turns out that we can write the chemical potential for any of the thermodynamics potentials in exactly the same way (holding the proper variables constant). 
\begin{equation}
     \mu_i = \left(\frac{\partial A}{\partial n_i}\right)_{T,V,n_j}, \qquad  \mu_i = \left(\frac{\partial U}{\partial n_i}\right)_{S,V,n_j}
\end{equation}
This result is not magic, simply write down the total differential for each of these thermodynamic potentials. 
The term you generate to account for different species is the same in each expression. 

\subsection*{Quantum Statistical Mechanics}
Let's start connecting back to statistical mechanics (see Lecture 4). 
We can cast our Chemical Potential and Free Energies in terms of the partition function. 
\begin{equation}
     \mu_i = -RT \ln\left(\frac{q_i}{N!}\right)
\end{equation}
Here q$_i$ refers to a molecules individual partition function and is a function of T and V. 
If we think way back, we started writing the partition function as a sum over energy states. 
\begin{equation}
\begin{split}
     q &= \sum_i e^{-\beta E_i}\\
     q &= q_{trans}q_{rot}q_{vib}q_{ele}
     \end{split}
\end{equation}
We can also write this partition function as a product over the individual degrees of freedom for an atom (the second line).
These 'Internal' degrees of freedom hold all of the chemistry that differentiates elements from one-another. 

These two statements are not consistent, the first line does not account for the chemical nature of the different atoms. 
Every atom type will have a different Zero-Point energy that we need to account for.
The equation above just sets this zero-point to 0, therefore the term would be 1 and we could just ignore it, but if we have different elements we need to start tracking them.
A more general statement would be (the last term accounts for the zero-point) 
\begin{equation}
    q = \sum_i e^{-\beta E_i}e^{-\beta E_{0,i}}
\end{equation}
We could then write our individual atoms partition function for 3-Dimensional space as
\begin{equation}
    q_i = \frac{V}{\Lambda^3}q_{trans}q_{rot}q_{vib}q_{ele}e^{-\beta E_{0,i}}
\end{equation}
With this generalized molecule partition function we can now calculate all of our variables. 
\begin{equation}
    \mu_i = -RT\ln\left(\frac{q_i}{N_i}\right)
\end{equation}
We then know at equilibrium we can constrain our reaction (dG=0).
\begin{equation}
    \begin{split}
        0 &= \sum_i\nu_i\mu_i\\
        0 &= -aRT\ln\left(\frac{q_A}{N_A}\right) + -bRT\ln\left(\frac{q_B}{N_B}\right) + cRT\ln\left(\frac{q_C}{N_C}\right) + dRT\ln\left(\frac{q_D}{N_D}\right) \implies \\
        0 &= \ln\left[\frac{\left(\frac{q_C}{N_C}\right)^c\left(\frac{q_D}{N_D}\right)^d}{\left(\frac{q_A}{N_A}\right)^a\left(\frac{q_B}{N_B}\right)^b}\right] \\
        1 &= \frac{\left(\frac{q_C}{N_C}\right)^c\left(\frac{q_D}{N_D}\right)^d}{\left(\frac{q_A}{N_A}\right)^a\left(\frac{q_B}{N_B}\right)^b} \\
        & \frac{N_C^c N_D^d}{N_A^a N_B^b} = \frac{q_C^c q_D^d}{q_A^a q_B^b}
    \end{split}
\end{equation}
So at equilibrium we can relate the partition functions of the individual atoms or molecules to their amounts. 

If we convert our number of particles (N) to concentrations (by dividing by V) we find
\begin{equation}
    \frac{\left(\frac{N_C}{V}\right)^c \left(\frac{N_D}{V}\right)^d}{
    \left(\frac{N_A}{V}\right)^a \left(\frac{N_B}{V}\right)^b} = \frac{\left(\frac{q_C}{V}\right)^c \left(\frac{q_D}{V}\right)^d}{\left(\frac{q_A}{V}\right)^a \left(\frac{q_B}{V}\right)^b} \equiv K(T)
\end{equation}
In statistical mechanics this expression is what we define to be the equilibrium coefficient. 
It is a function of Temperature (the partition function is a sum over energy, which is proportional to temperature). 
For our sample chemical reaction we could write
\begin{equation}
    \begin{split}
        K_{eq}(T) &= (\text{stuff})e^{-\beta(cE_{0,c} + dE_{0,d} - aE_{0,a} - bE_{0,b})}\\
        &= (\text{stuff}) e^{-\beta \Delta E_0}\\
        &\text{Graduate Stat. Mech.} \implies\\
        K_{eq}(T) &= e^{\Delta S_0 / R} \cdot e^{-\Delta E_0/RT}
    \end{split}
\end{equation}
 This result is somewhat interesting, recall A = U - TS from our fundamental equations. 
 This implies that 
 \begin{equation}
     K = e^{-\Delta A/RT}
 \end{equation}
 This is the end of Chapter 26 material, we are now going to return to the concept of mixing liquids. 
 
 \subsection*{Mixing Liquids}
 Think about space exploration, if you put liquid A in contact with a vacuum what happens?
 The liquid will boil... immediately.
 The molecules do not like beings super close, if there is no pressure pushing on them, they just spread out. 
 So take a container and place pure liquid A into it, evacuate the top and let the system then reach equilibrium. 
 At equilibrium our pure liquid and pure gas (after boiling) will obey 
 \begin{equation}
     \mu_L = \mu_g
 \end{equation}
We also know we can write the chemical potential as 
\begin{equation}
    \mu = \mu^0 + RT \ln\left(\frac{P}{P^0}\right)
\end{equation}
A natural question would be what reference should we use. 
If we chose the pure liquid for our reference we would have $\mu_l = \mu_l^*$ and $\mu_g = \mu_l +RT\ln\frac{P}{P*}$. 
At equilibrium the pressure would have to be equal to the pure for our example, (we have nothing else in the container except pure liquid A). 
Now consider two separate systems with two separate liquids A and B, each are placed in a container at vacuum, and then reach equilibrium making one container of pure A in both liquid and gas state, and the same for B. 
Now we take these two pure systems and mix them. 
We will assume both A and B are ideal, therefore the total pressure for the gas phase would follow Dalton's Law
\begin{equation}
    P_{tot} = P_A + P_B
\end{equation}
We can also use Rault's Law for the ideal solution. 
\begin{equation}
    P = x_AP_A^* + x_BP_B^* = x_AP_A^* + (1-x_A)P_B^*
\end{equation}

\section{Supplemental Notes}
Whenever we deal with mixtures, there are a few relationships we can use to determine things like pressure or concentration. 
It is important to realize that these relationships depend on the physical state of your systems (gases and liquids behave differently). 
This is a brief review of general chemistry topics that can be useful for actually solving problems. 

\subsection*{Dalton's Law of Partial Pressures}
When you have a mixture of gases within a container, we define the \textbf{Partial Pressure} of each gas within the mixture to be the pressure of the individual gas as if it were alone in the container.
From experiment Dalton concluded that the total pressure of a gas mixture is simply the sum of the partial pressures composing the mixture.
This is an approximation, it will be valid for moderate pressures and high temperatures. 
Consider an ideal gas
\begin{equation}
    P_{tot} = P_A + P_B + P_C + \cdots = (n_A + n_B + n_C + \cdots)\frac{RT}{V } = n_{tot}\frac{RT}{V}
\end{equation}
If we are only interested in a single component of the system we would write
\begin{equation}
    \frac{P_A}{P_{tot}} = \frac{n_A}{n_{tot}} \implies P_A = \frac{n_A}{n_{tot}}P_{tot} \implies P_A \equiv x_A P_{tot}
\end{equation}
Where we define the \textbf{Mole Fraction} x as the ratio of an individual atoms moles to the sum of the total moles in the system. 

\subsection*{Henry's Law}
When we have a mixture of a gas within a liquid it is not a surprise that the amount of gas that dissolves into the liquid depends on the pressure of the gas. 
Here the gas and liquid are different chemical species, the gas molecules buzz around the container, and eventually run into the liquid and enter the liquid phase.
The gas molecules are not technically becoming a liquid, but they interact with the liquid particles and dissolve into the liquid. 
So Henry's Law predicts the concentration of gas molecules that dissolve into a liquid, by making a linear approximation (valid for low concentrations or small mole fractions). 
\begin{equation}
    P_a = k x_a
\end{equation}
It simply says the partial pressure of gas a is proportional to the mole fraction of the gas, with some constant k.
The constant k will depend on both the solute and solvent (it captures the interactions occurring between the gas molecules and liquid molecules). 

\subsection*{Rault's Law}
Rault's Law is another linear relationship this time for solutions, telling us the vapor pressure of a solution is equal to the mole fraction times the pure component vapor pressure. 
\begin{equation}
    P_A = x_AP_A^0
\end{equation}
A solution that obeys Rault's law is called an \textbf{Ideal Solution}. 

\end{document}