\documentclass{article}
\usepackage[utf8]{inputenc}

\title{Chem-131C-Lec24}

\author{swflynn}
\date{June 2017}

\usepackage{natbib}
\usepackage{graphicx}
\usepackage{braket}
\usepackage{amsmath}
\usepackage[margin=0.7in]{geometry}
\usepackage{subfigure}
\usepackage{url}
\usepackage{float}
\usepackage[version=3]{mhchem}

\begin{document}

\maketitle

\section*{Lecture 24; 6/2/17}
Consider a container with pure A in both the liquid and vapour state. 
At equilibrium the chemical potential of the liquid and gas must be equal.
Our system only contains A (no other chemical species) so the chemical potential of A must be the pure state, $\mu_A^*$. 
If you were to open the container some of the liquid would evaporate because the partial pressure of the gas would decrease. 
Therefore $P_A < P_A^*$, the gas is at a lower pressure than the liquid, and therefore a lower chemical potential which would favor making more gas. 

\subsection*{Gibbs Free Energy of Mixing}
If we want to mix two different chemicals A and B, we will probably do it at constant temperature and pressure, therefore we are interested in the Gibbs Free Energy of Mixing. 
Imagine preparing two separate systems, one with pure A at equilibrium with its gas and liquid state, and the same for B. 
We know from previous lectures that we can write:
\begin{equation}
\begin{split}
    G &= \sum_i n_i \mu_i = n_A \mu_A + n_B \mu_B \\
    \mu_A &= \mu_A^* + RT \ln\left(\frac{P_A}{P_A^*}\right)
    \end{split}
\end{equation}
Initially we have two pure systems therefore 
\begin{equation}
    G_i = n_A\mu_A^* + n_B\mu_B^*
\end{equation}
After the two systems mix  we can write the pressure for the total system assuming ideal gases and solutions as 
\begin{equation}
P_{tot} = P_A + P_B = x_AP_A^* + x_BP_B^*
\end{equation}
The chemical potentials will no longer be that of the pure after mixing. 
\begin{equation}
\begin{split}
    G_f &= n_A\mu_A + n_B\mu_B \\
    &= n_A\left[\mu_A^* + RT \ln\left(\frac{P_A}{P_A^*}\right)\right] + n_B \left[\mu_B^* + RT \ln\left(\frac{P_B}{P_B^*}\right)\right] \\
    G_f &= n_A\mu_A^* + n_B\mu_B^* + n_ART\ln(x_A) + n_bRT\ln(x_B)
    \end{split}
    \end{equation}
    Taking the difference we find an expression for the change in Gibbs Free Energy due to mixing. 
\begin{equation}
\Delta_{mix}G = G_f - G_i = n_ART\ln(x_A) + n_BRT\ln(x_B)
\end{equation}

\subsection*{Ideal Solutions}
We just found the change in Gibbs Free Energy for mixing an ideal two component system. 
We can also write for constant temperature
\begin{equation}
\begin{split}
    G &= H - TS \\
    \Delta_{mix}G &=  \Delta_{mix}H - T\ \Delta_{mix}S
\end{split}
\end{equation}
For an ideal solution the enthalpy of mixing must be 0 (this is the definition of an ideal solution).
Therefore 
\begin{equation}
\Delta_{mix}S = -nR\left[x_A\ln(X_A) + X_B \ln(x_B)\right]
\end{equation}

An ideal solution assumes the interactions of A-A, B-B, and A-B are all equal in magnitude and nature. 
This is an approximation, it is not true in general (think of oil and water).
Engineering thermodynamics develops various models for these interactions in non-ideal cases. 
But for us we consider only ideal solutions, the mixing of which is entropically driven. 

\subsection*{Kinetic Theory of Gases}
The final topic in the course will be Chapter 27 of McQuarrie. 
Most all of the Statistical Mechanics we have discussed has been quantum statistical mechanics, relating distributions to the atomic partition functions. 
However, most statistical mechanics used in the chemistry community is Classical Statistical Mechanics. 
These methods use Newton's Equations to treat atoms like spheres during computer simulations.
These types of models can be used on large systems such as proteins and polymers (quantum mechanics calculations can only be done on $\approx$ 200 atoms, microsecond timescales). 
These simulations look at both the positions and velocities of the particles, therefore the \textbf{Phase Space} describing these systems must be 6-dimensional. 
This means our probability distribution in phase space is a function of six variables; P(x,y,x,p$_x$,p$_y$,p$_z$).
Here P is the probability distribution and p is the momentum. 
We can write our probability in a general manner in terms of the Hamiltonian (H) and a normalization q (this q is the classical statistical mechanical partition function). 
\begin{equation}
    P(x,y,z,p_x,p_y,p_z) = \frac{1}{q}e^{-\beta H(x,y,z,p_x,p_y,p_z)}
\end{equation}
We know the Hamiltonian can be broken into a sum over kinetic and potential energy, where the potential energy U is a function of space only U(x,y,z). 
For the particle in a box we defined the potential to be 0 within the box and infinite outside the box. 
Making the potential outside the box infinite makes the system classical stat. mech., if the energy is infinite there is no way for quantum tunneling to occur.

Recall the partition function is just a normalization, so we must integrate over our space (a 6-dimensional integral). 
\begin{equation}
    q = \idotsint dxdydzdp_xdpydp_z e^{\frac{1}{2\pi mkT}(p_x^2+p_y^2+p_z^2)}
\end{equation}
Here the exponential term is just the classical kinetic energy. 
Now recall from out quantum stat mech that q = $\frac{V}{\Lambda^3}$ which is dimensionless. 
We need to make our classical expression dimensionless too. 
To do this we will need to divide by a factor of h. 
I will not prove this here, but h is roughly the size of a quantum state (this of the uncertainty principle $\Delta x \Delta p \propto h$, and it turns out that $\frac{dxdp}{h}$ is dimensionless. 
Therefore the normalized integral generating our classical partition function needs a factor of $\frac{1}{h^3}$. 
\begin{equation}
    q = \idotsint \frac{dxdydzdp_xdpydp_z}{h^3} e^{\frac{1}{2\pi mkT}(p_x^2+p_y^2+p_z^2)}
\end{equation}
So this integral is a dimensionless quantity that gives the normalization over our classical statistical mechanical system.
We can note that 3 of the integrals run from 0 to $\infty$ and the other 3 run over the dimensions of our box; L$_x$, L$_y$, L$_z$. 
These integrals are separable, you will break them up over the spatial dimensions, and they can be evaluated as Gaussian integrals. 
I will not do the math here, but the final answer will be something like 
\begin{equation}
    \idotsint = \frac{V}{\left(\frac{h}{2\pi m kT}\right)^3} \propto \frac{V}{\Lambda^3}
\end{equation}
So that classical partition function is of the same dimensions as the quantum statistical mechanics partition function. 

\subsection*{Velocity Distribution}
We can also discuss the velocity distribution within our system. 
Notation is always a problem, I will remain with the notation Craig uses, now U$_x$ is the velocity in the x direction, do not confuse it with the potential energy. 
The velocity distribution depends on the 3 spacial dimensions velocities 
\begin{equation}
    F(U_x,U_y,U_z) = \left(\frac{m}{2\pi kT}\right)^{\frac{3}{2}}e^{\frac{m}{2kT}(U_x^2+U_y^2+U_z^2)}
\end{equation}
This integral is normalized therefore 
\begin{equation}
    \iiint F(U_x,U_y,U_z)dU_xdU_ydU_z = 1
\end{equation}
If we wanted to calculate the average value of the velocity in the x direction we would write down the standard integral. 
\begin{equation}
    \braket{U_x} = \iiint U_x \cdot F(U_x,U_y,U_z)dU_xdU_ydU_z
\end{equation}
If we then separate the function out we find that the integral of interest is really just 
\begin{equation}
    \int_{-\infty}^\infty U_x e^{-aU_x^2}dU_x = 0
\end{equation}
We know this integral is just 0 because the function is odd and we are integrating over all of space.
This answer should be intuitive, if the gas is sitting in a box, the box is not moving. 
Therefore the average velocity in any spatial direction has to be 0!

\end{document}